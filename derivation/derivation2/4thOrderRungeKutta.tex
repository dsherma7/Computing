%% go to the website http://en.wikibooks.org/wiki/LaTeX For Help 
\documentclass[11.5pt]{article}
%Math Related Packages
\usepackage{mathtools}
\usepackage{amsmath}
\usepackage{amsfonts}
\usepackage{amssymb}
\usepackage{relsize}
% Pseudocode Packages
\usepackage{algorithm}
\usepackage{algpseudocode}
\usepackage[colorinlistoftodos]{todonotes}
% Automata/Graph Packages
\usepackage{tikz}
\usepackage{pgf}
\usetikzlibrary{arrows,automata}
\usepackage[latin1]{inputenc}
\usetikzlibrary{automata,positioning}
%%%Formatting Options for Pages
%\usepackage[a4paper,left=2cm,right=2cm]{geometry} %Change Margins
\usepackage[colorlinks=true, allcolors=blue]{hyperref} % adds hyperlinks
\usepackage{listings}
\usepackage{multicol}
\usepackage{perpage}
\MakePerPage{footnote}
%%%Graphics Packages and Caption Tools
\usepackage{graphicx}
\usepackage{fullpage}
\newcounter{Figure} 
\setcounter{Figure}{1}
\usepackage{slashbox}
\usepackage{ulem}
\usepackage{csvsimple}
%%%Colors Package with Custom Defined Colors
\usepackage{color}
\usepackage{colortbl}
\definecolor{Gray}{gray}{0.9}
\definecolor{codegreen}{rgb}{0,0.6,.2}
\definecolor{codegray}{rgb}{0.5,0.5,0.5}
\definecolor{codepurple}{rgb}{0.58,0,0.82}
\definecolor{backcolor}{rgb}{0.94,0.94,1}
\definecolor{light-gray}{gray}{0.55}
%%Custom Functions and Commands
\newcommand\indentFour{\indent\indent\indent\indent}
\newcommand\indentThree{\indent\indent\indent}
\newcommand\indentTwo{\indent\indent}
\newcommand\tab{\ \ \ }
\newcommand\proof{\setcounter{equation}{0}\hfill\fbox{\rule{.02in}{0pt}\rule[0ex]{0pt}{.5ex}}}
%%Color Commands
\newcommand\BLK{\color{black}}
\newcommand\GRY{\color{Gray}}
\newcommand\BLU{\color{blue}}
\newcommand\blu{\color{CornflowerBlue}}
\newcommand\RED{\color{red}}
\newcommand\red{\color{RubineRed}}
\newcommand\ORG{\color{BurntOrange}}
\newcommand\org{\color{Peach}}
\newcommand\GRN{\color{ForestGreen}}
\newcommand\grn{\color{LimeGreen}}
\newcommand\PRP{\color{RoyalPurple}}
\newcommand\prp{\color{DarkOrchid}}
%%Math Commands
\newcommand{\Lbra}{\left\langle}
\newcommand{\Rbra}{\right|}
\newcommand{\Rket}{\right\rangle}
\newcommand{\Lket}{\left|}
\newcommand{\bra}[1]{\Lbra #1 \Rbra}
\newcommand{\ket}[1]{\Lket #1 \Rket}
\newcommand{\braket}[1]{\Lbra #1 \Rket}
\newcommand{\Exists}{\ \exists \ }
\newcommand{\Forall}{\ \forall \ }
\newcommand{\abs}[1]{\left| #1 \right|}
\newcommand{\Frac}[2]{\left(\frac{#1}{#2}\right)}
\newcommand{\Mat}[1]{\left[\begin{matrix} #1 \end{matrix}\right]}
\newcommand{\vhi}{\varphi}
\newcommand{\R}{\ \mathbb{R}}
\newcommand{\C}{\ \mathbb{C}}
\newcommand{\N}{\ \mathbb{N}}
\newcommand{\Z}{\ \mathbb{Z}}
\newcommand{\I}{\ \mathbb{R/Q}}
\newcommand{\x}{\mathrm{x}}
\newcommand{\mbf}[1]{\mathbf{#1}}
\newcommand{\Dpart}[2]{\frac{\partial #1}{\partial #2}}
\DeclarePairedDelimiter{\ceil}{\lceil}{\rceil}
\newcommand{\U}{\underline}
%Numbering
\newcounter{graphics}
\newcounter{tables}
%For coding
\lstdefinestyle{small}{
    backgroundcolor=\color{backcolor},   
    commentstyle=\color{codegreen},
    keywordstyle=\color{blue},
    numberstyle=\tiny\color{codegray},
    stringstyle=\color{codepurple},
    basicstyle=\footnotesize,
    breakatwhitespace=false,         
    breaklines=false,                 
    captionpos=b,                    
    keepspaces=false,                 
    numbers=left,                    
    numbersep=5pt,                  
    showspaces=false,                
    showstringspaces=false,
    showtabs=false,                  
    tabsize=4
}
%\lstinputlisting[language=Matlab]{cub_spline.m} %Use this to add code
\renewcommand{\abstractname}{Abstract}
\begin{document}
\title{MAT 125B - Homework \# 7}
\author{Douglas Sherman \#: 913348406}
\date{January 27th, 2017}
\maketitle
\rule{\textwidth}{1pt}
\lstset{style=small} %set code style, see preamble


Consider the initial value problem given by
$$ \left\lbrace \begin{matrix}
y'(t) &= f(t,y(t))\\
y(t_0) &= y_0
\end{matrix}\right.$$
for a function Lipschitz continuous function $f$. To approximate the solution $y(t)$, the Runge-Kutta method uses an intermediate $y$ value to enhance the approximation of each successive approximation $y_k \approx y(t_k)$. By using multiple intermediate steps we can define a 4th order Runge Kutta method as
\begin{align*}
y_{k+\alpha} &= \alpha f(t_k,y_k)\\
y_{k+\beta} &= \beta f(t_k + \alpha h,y_{k+\alpha})\\
y_{k+\gamma} &= \gamma f(t_k + \beta h,y_{k+\beta})\\
y_{k+\delta} &= \delta f(t_k + \gamma h,y_{k+\gamma})\\
y_{k+1} &= y_k + \eta h l
\end{align*}
for some $\alpha, \beta,$ and $\gamma$ that must be consistent with the Taylor expandion of $y' = f$. Define $f_{y}$ and $f_{t}$ as the partial derivatives of $f$ with respect to $y$ and $t$ respectively. Moreover, since $f$ is continuously differentiable, the mixed partials of $f$ are equivalent. Then the chain rule and Taylor's theorem gives us
\begin{align*}
f(t_k+\alpha h,\widetilde{y}_{k+\alpha}) &= f(t_k+\alpha h,y_k + \alpha h f(t_k,y_k))\\
&= f(t_k,y_k) +  \alpha h \left[ f_t + ff_y\right] + \frac{(\alpha h)^2}{2} \left[f_{tt} + ff_{ty} + f^2f_{yy} \right] + \mathcal{O}(h^3)\\
f(t_k+\gamma h,\widetilde{y}_{k+\gamma}) &= f(t_k+\gamma h, y_k + \gamma h f(t_{k}+\alpha h,\widetilde{y}_{k+\alpha}))\\
&= f(t_k,y_k) + \gamma (\gamma hf(t_k+\alpha h,\widetilde{y}_{k+\alpha))\left( \right)

\end{align*}
Thus the iterative step for $y_{k+1}$ is given by
\begin{align*}
y_{k+1} &= y_k + \eta h f(t_k,y_k) + \nu h f(t_k + \gamma h, \widetilde{y}_{k+\gamma})\\
&= y_k + \eta h f(t_k,y_k) + \nu h \left( f(t_k,y_k) +  \gamma h \left[ f_t + ff_y\right] + \frac{(\alpha h)^2}{2} \left[f_{tt} + ff_{ty} + f^2f_{yy} \right] + \mathcal{O}(h^3)\right)
\end{align*}
Moreover, the Taylor expansion of the actual solution $y(t_{k+1})$ is given by
\begin{align*}
y(t_{k+1}) &= y(t_k) + hy'(t_k) + \frac{h^2}{2}y''(t_k) + \frac{h^3}{6}y'''(t_k) + \frac{h^4}{24}y^{(4)}(t_k) + \frac{h^5}{120}y^{(5)}(t_k) + \mathcal{O}(h^6)\\
&= y(t_k) + hf + \frac{h^2}{2}\left[f_t + ff_y \right] + \frac{h^3}{6}\left[f_{tt} + ff_{ty} + f^2f_{yy} \right] + \frac{h^4}{24}\left[ f_{ttt} + 3ff_{tty} + 3f^2f_{tyy} + f^3f_{yyy}\right]\\ &\quad + \frac{h^5}{120}\left[f_{tttt} + 4ff_{ttty} + 6f^2f_{ttyy} + 4f^3f_{tyyy}+f^4f_{yyyy} \right] + \mathcal{O}(h^6)
\end{align*}
Hence, we must choose $\alpha,\beta,\gamma,\delta,$ and $\eta$ so that $y(t_{k+1}) = y_{k+1}$. Since, the seperate terms are independent the corresponding coefficients of $h$ must be equal. Hene we obtain a system of equations given by
\begin{align*}
\beta + \gamma + \eta &= 1 & \BLU h^0 \BLK\\
(\gamma+\eta)(\alpha +\delta)&= \frac{1}{2} & \BLU h^1 \BLK\\
(\gamma + \eta)(\alpha^2+\delta^2) &= \frac{1}{3} & \BLU h^2 \BLK\\
(\gamma +\eta)(\alpha^3+\delta^3)  &= \frac{1}{4} & \BLU h^3 \BLK\\
(\gamma +\eta)(\alpha^4+\delta^4) &= \frac{1}{5} & \BLU h^4 \BLK\\
\end{align*}
Since we have $5$ variables, we require all 5 of the above equations to solve for each. However, an obvious solution is 




\end{document}